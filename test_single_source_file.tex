%\documentclass[11pt]{article}
\documentclass[12pt]{article}
%\usepackage{booktabs}
\usepackage{ctable}
\usepackage{a4wide}
\usepackage[T1]{fontenc}
\usepackage[utf8]{inputenc}
\usepackage{fouriernc}
\usepackage{longtable}
\usepackage{marginnote}


\newcommand{\oak}[1]{{\leavevmode\color{red}#1}}
\newcommand{\nvf}[1]{{\leavevmode\color{red}#1}}

\makeatletter
\def\input@path{{source_files/}}
%or: \def\input@path{{/path/to/folder/}{/path/to/other/folder/}}
\makeatother

% I use the various longtable formats to read a piece of text in just
% one language. This makes it easier to catch typo's and bad
% sentences.

\newcolumntype{O}{>{\raggedright\arraybackslash}p{15cm}} % only column
\newcolumntype{H}{>{\setbox0=\hbox\bgroup}c<{\egroup}@{}} % hide
\newcolumntype{R}{>{\raggedleft\arraybackslash}p{7.5cm}}
\newcolumntype{L}{>{\raggedright\arraybackslash}p{7.5cm}}

\begin{document}
\tableofcontents
%\begin{longtable}{LLLL} % all
\begin{longtable}{L|L|L} % all
%\begin{longtable}{OHH} % just dutch
%\begin{longtable}{HOH} % just english
%\begin{longtable}{LLH} % dutch english
%\begin{longtable}{HHO} % just turkish
%\begin{longtable}{HLL} % english turkish
%\begin{longtable}{L||HR} % english turkish
\toprule
<en>A Wise Counting
<nl>Slim geteld
<tr>Zekice bir sayım
<en>Emperor Akbar was in the habit of putting riddles and puzzles to his courtiers. 
<nl>Keizer Akbar had de gewoonte raadsels en puzzels aan zijn hovelingen voor te leggen. 
<tr>İmparator Akbar'ın saray adamlarına bilmece ve bulmaca sorma alışkanlığı vardı. 
<en>He often asked questions which were strange and witty. 
<nl>Hij vroeg vaak vragen die vreemd waren en geestig.
<tr>Çoğunlukla tuhaf ve nükteli sorular sorardı.
<en>It took much wisdom to answer these questions. 
<nl>Het vroeg veel wijsheid deze vragen te beantwoorden.
<tr>Bu sorulara cevap vermek çok akıllı olmayı gerektirirdi.
<en>Once he asked a very strange question. 
<nl>Eens vroeg hij een hele rare vraag.
<tr>Bir seferinde çok tuhaf bir soru sordu.
<en>The courtiers were dumb folded by his question. 
<nl>De hovelingen waren met stomheid geslagen door zijn vraag.
<tr>Saray adamları onun sorusu karşısında aptallaşmıştı.
<en>Akbar glanced at his courtiers. 
<nl>Akbar keek naar zijn hovelingen.
<tr>Akbar saray adamlarına baktı.
<en>As he looked, one by one the heads began to hang low in search of an answer. 
<nl>Terwijl hij keek, begon het ene na het andere hoofd te hangen op zoek naar een antwoord.
<tr>Bakarken, cevabı arayan başlar birer birer eğilmeye başladı.
<en>It was at this moment that Birbal entered the courtyard. 
<nl>Precies op dat moment  betrad Birbal de hoftuin.
<tr>İşte tam bu anda Birbal avluya girdi.
<en>Birbal who knew the nature of the emperor quickly grasped the situation and asked, 
<nl>Birbal, die het karakter van de keizer kende, doorzag de situatie snel en vroeg,
<tr>İmparatorun huyunu bilen Birbal durumu hemen anladı ve sordu,
<en>"May I know the question so that I can try for an answer". 
<nl>`Mag ik de vraag weten, zodat ik een antwoord kan proberen te vinden?'
<tr>"Soruyu öğrenebilir miyim, böylece ben de cevabı bulmayı deneyebilirim".
<en>Akbar said, "How many crows are there in this city?" 
<nl>Akbar zei: `Hoeveel kraaien zijn er in deze stad?'
<tr>Akbar dedi ki, "Bu şehirde kaç tane karga var?"
<en>Without even a moment's thought, Birbal replied "There are fifty thousand five hundred and eighty nine crows, my lord".
<nl>Zonder zelfs maar een moment na te denken, antwoordde Birbal: `Er zijn vijftigduizend vijfhonderd en negen-en-tachtig kraaien, mijn heer.'
<tr>Birbal bir an bile düşünmeden cevap verdi "Elli bin beş yüz seksen dokuz" lordum.
<en>"How can you be so sure?" asked Akbar. 
<nl>`Hoe weet je dat zo zeker?', vroeg Akbar. 
<tr>"Nasıl bu kadar emin olabilirsin?" diye sordu Akbar.
<en>Birbal said, "Make your men count, My lord. If you find more crows it means some have come to visit their relatives here. If you find less number of crows it means some have gone to visit their relatives elsewhere". 
<nl>Birbal zei: `Laat uw mannen tellen, mijn heer. Als u meer kraaien vindt, betekent het dat sommige gekomen zijn om hier  hun familie te bezoeken. Als u minder kraaien vindt, betekent het dat sommige bij hun familie elders op bezoek zijn.'
<tr>Birbal "Adamlarınız saysın lordum. Eğer daha fazla karga bulursanız, bu bazıları buraya akrabalarını ziyaret etmek için gelmiş demektir. Eğer daha az sayıda karga bulursanız, bu bazıları akrabalarını ziyaret etmeye başka yere gitmiş demektir" dedi.
<en>Akbar was pleased very much by Birbal's wit. 
<nl>Akbar was erg ingenomen met Birbals gevatheid.
<tr>Akbar Birbal'in kıvrak zekasından memnun kaldı.

\end{longtable}

\end{document}
%%% Local Variables:
%%% mode: latex
%%% TeX-master: t
%%% End:
