%\documentclass[11pt]{article}
\documentclass[12pt]{article}
%\usepackage{booktabs}
\usepackage{ctable}
\usepackage{a4wide}
\usepackage[T1]{fontenc}
\usepackage[utf8]{inputenc}
\usepackage{fouriernc}
\usepackage{longtable}
\usepackage{marginnote}


\newcommand{\oak}[1]{{\leavevmode\color{red}#1}}
\newcommand{\nvf}[1]{{\leavevmode\color{red}#1}}

\makeatletter
\def\input@path{{source_files/}}
%or: \def\input@path{{/path/to/folder/}{/path/to/other/folder/}}
\makeatother

% I use the various longtable formats to read a piece of text in just
% one language. This makes it easier to catch typo's and bad
% sentences.

\newcolumntype{O}{>{\raggedright\arraybackslash}p{15cm}} % only column
\newcolumntype{H}{>{\setbox0=\hbox\bgroup}c<{\egroup}@{}} % hide
\newcolumntype{R}{>{\raggedleft\arraybackslash}p{7.5cm}}
\newcolumntype{L}{>{\raggedright\arraybackslash}p{7.5cm}}

\begin{document}
\tableofcontents
%\begin{longtable}{LLLL} % all
\begin{longtable}{L|L|L} % all
%\begin{longtable}{OHH} % just dutch
%\begin{longtable}{HOH} % just english
%\begin{longtable}{LLH} % dutch english
%\begin{longtable}{HHO} % just turkish
%\begin{longtable}{HLL} % english turkish
%\begin{longtable}{L||HR} % english turkish
\toprule
Slim geteld&
A Wise Counting&
\\
Keizer Akbar had de gewoonte raadsels en puzzels aan zijn hovelingen voor te leggen. &
Emperor Akbar was in the habit of putting riddles and puzzles to his courtiers. &
\\
Hij vroeg vaak vragen die vreemd waren en geestig.&
He often asked questions which were strange and witty. &
\\
Het vroeg veel wijsheid deze vragen te beantwoorden.&
It took much wisdom to answer these questions. &
\\
Eens vroeg hij een hele rare vraag.&
Once he asked a very strange question. &
\\
De hovelingen waren met stomheid geslagen door zijn vraag.&
The courtiers were dumb folded by his question. &
\\
Akbar keek naar zijn hovelingen.&
Akbar glanced at his courtiers. &
\\
Terwijl hij keek, begon het ene na het andere hoofd te hangen op zoek naar een antwoord.&
As he looked, one by one the heads began to hang low in search of an answer. &
\\
Precies op dat moment  betrad Birbal de hoftuin.&
It was at this moment that Birbal entered the courtyard. &
\\
Birbal, die het karakter van de keizer kende, doorzag de situatie snel en vroeg,&
Birbal who knew the nature of the emperor quickly grasped the situation and asked, &
\\
`Mag ik de vraag weten, zodat ik een antwoord kan proberen te vinden?'&
"May I know the question so that I can try for an answer". &
\\
Akbar zei: `Hoeveel kraaien zijn er in deze stad?'&
Akbar said, "How many crows are there in this city?" &
\\
Zonder zelfs maar een moment na te denken, antwoordde Birbal: `Er zijn vijftigduizend vijfhonderd en negen-en-tachtig kraaien, mijn heer.'&
Without even a moment's thought, Birbal replied "There are fifty thousand five hundred and eighty nine crows, my lord".&
\\
`Hoe weet je dat zo zeker?', vroeg Akbar.& 
"How can you be so sure?" asked Akbar. &
\\
Birbal zei: `Laat uw mannen tellen, mijn heer. Als u meer kraaien vindt, betekent het dat sommige gekomen zijn om hier  hun familie te bezoeken. Als u minder kraaien vindt, betekent het dat sommige bij hun familie elders op bezoek zijn.'&
Birbal said, "Make your men count, My lord. If you find more crows it means some have come to visit their relatives here. If you find less number of crows it means some have gone to visit their relatives elsewhere". &
\\
Akbar was erg ingenomen met Birbals gevatheid.&
Akbar was pleased very much by Birbal's wit. &
\\

\end{longtable}

\end{document}
%%% Local Variables:
%%% mode: latex
%%% TeX-master: t
%%% End:
