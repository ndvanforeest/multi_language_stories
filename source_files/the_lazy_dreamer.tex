De luie dromer&
The Lazy Dreamer&
\\
Eens, in een klein dorp, woonde een arme Brahmaan.&
Once, in a small village, there lived a poor Brahmin. &
\\
Hij was erg geleerd, maar deed de hele dag niets.&
He was very learned, but did nothing all day. &
\\
Hij leefde van de aalmoezen die de dorpelingen hem elke dag gaven.&
He lived on the alms the villagers gave him every day. &
\\
Op een dag, zoals gewoonlijk, stond de Brahmaan op in de ochtend, voerde zijn ochtendrituelen uit, en ging naar buiten om te bedelen om aalmoezen.&
One day, as usual, the Brahmin got up in the morning, performed his morning rituals and set out to beg for alms. &
\\
Terwijl hij van deur tot deur ging, gaven mensen hem verschillende dingen. &
As he went from door to door, people gave him several things. &
\\
Sommigen gaven linzen. Anderen gaven hem rijst, en weer anderen gaven hem groenten.&
Some gave dal. Others gave him rice and yet others gave him vegetables. &
\\
Maar \'e\'en ruimhartige  dame gaf de Brahmaan een grote hoeveelheid meel.&
But one generous lady gave the Brahmin a large measure of flour. &
\\
`Ah, wat een geluk. I hoef voorlopig niet te bedelen om aalmoezen', dacht de Brahmaan bij zichzelf.&
“Ah! What good luck. I will not have to beg for alms for a long time," thought the Brahmin to himself. &
\\
Hij ging naar huis en kookte zijn lunch.&
He went home and cooked his lunch. &
\\
Nadat hij gegeten had, deed de Brahmaan het meel in een grote aarden pot en hing het bij zijn bed.&
After he had eaten, the Brahmin put the flour into a large mud pot and hung it near his bed. &
\\
`Nu zal het veilig zijn voor de ratten', zei hij bij zichzelf toen hij ging liggen in zijn bed voor een middagslaapje.&
“Now, it will be safe from rats," he said to himself as he lay down in his cot for an afternoon nap. &
\\
Hij begon te denken, `Ik wil dit meel bewaren totdat er een hongersnood is. &
He began to think, “I will save this flour until there is a famine. &
\\
Dan zal ik het verkopen voor een heel goede prijs.&
Then I will sell it at a very good price. &
\\
Daarmee zal ik een paar geiten kopen.& 
With that, I will buy a pair of goats. &
\\
Al snel zal ik een grote kudde geiten hebben.&
Very soon, I will have a large flock of goats. &
\\
Met hun melk zal ik geld verdienden.&
With their milk, I will make more money. &
\\
Dan zal ik een koe kopen en een stier.&
Then I will buy a cow and a bull. &
\\
Al snel zal ik ook een grote kudde koeien hebben.&
Very soon I will also have a large herd of cows. &
\\
Hun melk zal me veel geld bezorgen.&
Their milk will fetch me a lot of money. &
\\
Ik zal erg rijk worden.&
I will become very wealthy. &
\\
Ik zal voor mezelf een paleis bouwen en trouwen met een mooie vrouw.&
I will build for myself, a huge palace and get married to a beautiful woman.&
\\
Dan zullen we een kleine zoon hebben.&
 Then we will have a little son. &
\\
Ik zal een trotse vader zijn.&
I will be a proud father. &
\\
Binnen een paar maanden zal mijn zoon beginnen te kruipen.&
In a few months my son will start crawling. &
\\
Hij zal stout zijn and ik zal me zorgen maken of hij in problemen zal komen.&
He will be mischievous and I will be very worried that he may come to some harm. &
\\
Ik zal mijn vrouw roepen om voor hem te zorgen&
I will call out to my wife to take care of him. &
\\
Maar zij zal druk zijn met huishoudelijk werk, en mijn roep negeren.&
But she will be busy with house work and will ignore my call. &
\\
Ik zal zo boos worden dat ik haar zal trappen om haar een lesje te leren.'&
I will get so angry, I will kick her to teach her a lesson like this.'&
\\
De Brahmaan gooide zijn been omhoog.&
The Brahmin threw out his leg up. &
\\
Zijn voet raakte de pot met meel boven zijn hoof en die kwam naar beneden met doorslaande knal, het meel helemaal over de vieze grond verkwistend.&
His foot hit the pot of flour hanging overhead and it came down with a resounding crash, spilling the flour all over the dirty floor. &
\\
De luie Brahmaan realiseerde dat zijn domheid en trots hem een waardevolle hoeveelheid meel hadden gekost.&
The lazy Brahmin realised that his foolishness and vanity had cost him a precious measure of flour.&
\\
De luiheid en domheid leerden hem een les.&
 The laziness and foolishness taught him a lesson. &
\\
Daarna leefde hij een actief leven dat veel opleverde.&
Thereafter he lived an active life which took to heights. &
\\

