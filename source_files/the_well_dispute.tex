Het geschil over de put&
The Well Dispute&
\\
Eens was er een klacht bij de koning Akbars rechtbank.&
Once there was a complaint at King Akbar's court.&
\\
Er waren twee buren die hun tuin deelden.&
There were two neighbors who shared their garden.&
\\
In die tuin was er een bron die in eigendom was van Iqbal Khan.&
In that garden, there was a well that was possessed by Iqbal Khan.&
\\
Zijn buurman, die boer was, wilde de put kopen voor irrigatiedoeleinden.&
His neighbor, who was a farmer, wanted to buy the well for irrigation purposes.&
\\
Daarom tekenden ze een overeenkomst, waarna de boer de bron bezat.&
Therefore they signed an agreement between them, after which the farmer owned the well.&
\\
Maar zelfs na de verkoop van de put aan de boer, bleef Iqbal water halen uit de put.&
Even after selling the well to the farmer, Iqbal continued to fetch water from the well.&
\\
Boos hierover, was de boer naar de rechtbank gekomen om zijn recht te halen bij  koning Akbar.&
Angered by this, the farmer had come to the court to get justice from King Akbar.&
\\
Koning Akbar vroeg Iqbal naar de reden om water uit de bron te halen, zelfs nadat hij die aan de boer had verkocht.&
King Akbar asked Iqbal the reason for fetching water from the well even after selling it to the farmer.&
\\
Iqbal antwoordde dat hij alleen de bron aan de boer had verkocht, maar niet het water erin.&
Iqbal replied that he had sold only the well to the farmer but not the water inside it.&
\\
Koning Akbar wilde dat Birbal, die aanwezig was in de rechtbank en naar het probleem luisterde, het geschil zou oplossen.&
King Akbar wanted Birbal who was present in the court listening to the problem, to solve the dispute.&
\\
Birbal kwam naar voren en gaf een oplossing.&
Birbal came forward and gave a solution.&
\\
Hij zei: "Iqbal, je zegt dat je alleen de bron aan de boer hebt verkocht.&
He said "Iqbal, You say that you have sold only the well to the farmer.&
\\
En je beweert dat het water van jou is.&
And you claim that the water is yours.&
\\
Hoe komt het dan dat je jouw water in iemand anders put kunt houden zonder huur te betalen?'&
Then how come you can keep your water inside another person's well without paying rent?"&
\\
Iqbal's bedrog werd op een slimme manier weerlegd.&
Iqbal's trickery was countered thus in a tricky way.&
\\
De boer kreeg zijn recht en Birbal werd fatsoenlijk beloond.&
The farmer got justice and Birbal was fairly rewarded.&
\\
