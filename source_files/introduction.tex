
\section{Introduction and guide}
\label{sec:introduction-guide}

Some time ago I took up the challenge to learn Turkish. For me it was a completely new language, and I was just curious to see how I would fare, and how long it would take to train my internal neural network. To embark on this project I joined forces with my colleague Onur, a Turkish guy, who happened to have to learn Dutch. So we made a few plans on how to attack the problem of learning a new language.

We decided not to read grammar books, or books written for language learners. As kids we did use any such book to learn our language for the first 10 year or so. Moreover, as a deterrent, such books are nearly always (very) boring, full of stories we didn't care to read of know. And worse, some of these books simply teach the wrong things. Consider the following  conversation, an example which I got from some actual `Improve your Dutch' book, written for English speakers. Translated to English the conversation roughly went like this. A guy enters a bookshop: `Good morning.' `Good morning sir. How can I help you?' `I'm looking for a book, a birthday present for  my six-year old son.' `What type of book might your son like?' `I don't give a s---. You are supposed to know this.' Lo and behold! My eyes must  be betraying me. This can't be a beginner's book, can it? This is definitely not the way we Dutch communicate. Now, as a native Dutch speaker, I  just know  this, but, in Turkish, I have not clue what is appropriate or not. However, I certainly don't aim to say such things unknowingly  in any Turkish conversation. 

%So, Onur and I decided to leave such books unread. 

Besides that, there are often technical problems with the layout of these books. Often, the sentences, in my case Turkish, are interspersed with  English sentences that serve as parallel translation. This is very practical in the beginning, when studying the story line for line. However,  this format makes it hard to cover the English sentences when studying the Turkish text again. Worse yet,  reading the Turkish in one stretch becomes nearly impossible, because I  constantly have to hop over the English sentences. Like this, I can't train my brain into `thinking in Turkish'. 

So, Onur and I decided to do things differently. Our first idea was to take the 100 most frequently used Dutch words, and then translate then, via English---our `stone of Rosetta'. This appeared to make sense, but we soon discovered it wasn't such a good idea after all. It turns out that many of these words are used to give subtle shifts, shades if you will, of meaning to the sentences. (As a native I simply didn't realize  this up front.) So we stopped with this after a few days. As long as one doesn't know the word `cow' in some language, why first learn words like `How are you, old chap?' `I am very well indeed'. 

Then we started to use parallel texts, but this also turned out to be a, by and large, disappointing experience: these parallel text appear to made to show off the literacy of the translator,  not meant to help a person totally unfamiliar with the language. Here is an example I found in an English-Dutch children's book. In English: ‘Bob, are you my best friend or are you best friend of that bloody rascal?’ Then  its Dutch translation (but translated back to English): `Bob, who is your best friend, John or me?' So, if I wouldn't know Dutch, this translation would have left me totally clueless about its meaning. Actually, I am totally perplexed about why, in the first place, a translator, when making a \emph{parallel} text, does not aim to  stick to the very word sequence of the original to the extent possible, and second, try to be as literal as possible. If I were to know both languages, I wouldn't be reading such books anyway. But, given that these books are intended (as I suppose) for people that don't know both languages, then why mess up the text to the extent one can't possbiby retrieve one sentence from the other? 

So, all in all, given that the material available to us was pretty useless, for our purposes at least, we  decided to take our fate in our hands. This list of stories is the result. We took simple Enlish stories from the web and wrote some ourselfs, and translated them to Dutch and Turkish. We used the following criteria for selecting the stories. 

The simple ones, appearing at the beginning of the book and which we mostly wrote,  had to be very concrete and vivid, and relate directly to daily experience.  Memorizing new words like this seems to be the easiest; one can nearly `feel the word'. Moreover, the simple stories should consist of really simple senctences. The shorter the better. To explain, Onur and I both have children, and  our children like to have read to them the same story over and over, at least ten times. Now that both of us are also learning a completely new language, we noticed that such simple repetition is  also very useful to us. All in all, it is hard to acquire a new language. Hence, sentences should be short, changes should be small, and there should be plenty of repetition, just to get a feel for how the language is used. `I am picking up a glass', `I am picking up a plate', and so on. One or two new words per sentence, not more, if possible. 

The more difficult stories, appearing towards the end, should be fun to read, so that we wouldn't mind studying and reading them a few times. The final story, `The Story of the Baked Head',  is from a book of J.J. Morier. It is truly funny, the use of  English language is great, but it is also very hard. Once one can read this, one can read most of the target language. 

In the translations from English to Dutch we used Google translate heavily, to do most of the heavy lifting. It is a fantastic piece of work, truely amazing how much it gets right. Once Google translate produced a rough first version, we went over the translations, and polished the text where necessary. For this we used two criteria, in order of importance. First, we aimed to keep the translation as literal as possible,  both in word sequence and meaning. Only if this would run counter to the fluency of the translation we would change the word sequence or actual wording. As an example, take the English sentence, "I am taking off my shoes.". In Dutch we say, literally, "I take off my shoes.". Translating is not a `one-to-one thing', and one should not force one language into the structure of the other.

\url{http://ingilizcebankasi.com/ingilizce-turkce-hikayeler/}
\url{http://www.english-for-students.com/}
\url{http://childhoodreading.com/}
\url{http://worldstories.org.uk/stories/} Very bad english
\url{http://genkienglish.net}
\url{http://eslyes.com/easyread/}
\url{http://literatureproject.com/arabian-nights/index.htm}
